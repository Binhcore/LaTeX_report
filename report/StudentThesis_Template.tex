\documentclass[ITR,BA,english,intermediate,tutorial]{LSR_thesis} 
\graphicspath{{pics/}}

%% Options:
%% LSR: LSR Template with Prof. Buss as default
%% ITR: ITR Template with Prof. Hirche as default

%% BA: Bachelorarbeit / Bachelor thesis
%% MA: Masterarbeit / Master thesis
%% HS: Hauptseminar / Scientific seminar
%% PP: Projektpraktikum / practical course
%% IP: Ingenieurpraxis
%% FP: Forschungspraxis
%% SeA: Semesterarbeit (MW)

%% english
%% german

%% final
%% intermediate

%% tutorial -> remove flag such that help files and todos are removed

%%% last changes: 7.02.2019 (v.gabler@tum.de)

%------------------------------------------------%
%_________CUSTOMIZE LATEX ONLY IN customize.tex!_% 
%_________ DO NOT MODIFY THE TEMPLATE!___________%
%________________________________________________%
% add customize first so you can access your commands in gloss, or fuse both into one file
\input{./include/customize.tex} % add custom commands etc in this file 
\input{./include/gloss.tex}		% add your glossary in this file 


%_______Start_Document______________________________________
\begin{document}
%%%%%%%%%%%%%%%%%%%%%%%%%%%%%%%%%%%%%%%%%%%%%%%%%%%%%%%%%%%%%%%
%%%%%%%%%%%%%%%%%%% title page %%%%%%%%%%%%%%%%%%%%%%%%%%%%%%%%
%%%%%%%%%%%%%%%%%%%%%%%%%%%%%%%%%%%%%%%%%%%%%%%%%%%%%%%%%%%%%%%
%% for an english theis. The title:
\title{Origami-inspired Muscles for a Soft Robotic Manipulator}
%% Für deutsche Arbeiten: Deutscher Titel:
%\title{Die Antwort auf Alles und Mehr - Ein Trauerspiel in 4 Akten}
% and English translation (optional)
%\titletranslation{Die Verwendung eines Deutschen Untertitels obliegt der Verantwortung der entsprechenden Betreuer}
% data about YOU!:
\student{An Binh Vu} 			%% your name
\studtitle{B.Sc.} 			%% Bachelor of Arts, Dr.~phil, etc.
\street{Sauerbruchstrasse 61}			%% your address
\city{81377 Munich}								%		"
\phone{+49 17645668868}			%% your telephone-no.

%% if more students are involved (e.g. PP) 
%--the following parted is not tested ---
% please report bugs to v.gabler@tum.de 
%\studenttwo{Zweiter Student}
%\studtitletwo{} 
%\studentthree{} 
%\studtitlethree{} 
%\studentfour{} 
%\studtitlefour{} 
%-----------------------------------------
\supervisor{Dr.-Ing. Stefan Sosnowski}			%% your supervisor
\finalrep{21.08.2019}						%% final presentation / date

\maketitle
%%%%%%%%%%%%%%%%%%%%%%%%%%%%%%%%%%%%%%%%%%%%%%%%%%%%%%%%%%%%%%%
%%%%%%%%%%%%%%%%%%%%%%%%%%%%%%%%%%%%%%%%%%%%%%%%%%%%%%%%%%%%%%%

\newpage
\cleardoublepage
\ifLSRITRtutorial
		\phantom{u}
		\phantom{1}\vspace{6cm}
	\begin{center}
		\add[inline]{In your final hardback copy, replace this page with the signed exercise sheet.}
		\vspace{3cm}
		\todo[inline,color=red!70]{Before modifying this document, READ THE INSTRUCTIONS AND GUIDELINES!}
	\end{center}
\else
	% in case you want to add the PDF directly add the task description in the include directory
	\includepdf[pages=1]{./include/task_desc.pdf}
\fi
\newpage

%%%%%%%%%%%%%%%%%%%%%%%%%%%%%%%%%%%%%%%%%%%%%%%%%%%%%%%%%%%%%%%
%%%%%%%%%%%%%%%%%%%%% abstract %%%%%%%%%%%%%%%%%%%%%%%%%%%%%%%%
%%%%%%%%%%%%%%%%%%%%%%%%%%%%%%%%%%%%%%%%%%%%%%%%%%%%%%%%%%%%%%%
\topmargin5mm
\textheight220mm
\pagenumbering{arabic}
\phantom{u}
\begin{abstract}
This thesis deals with building and controlling a finger-like actuator made out of thermoplastic polyurethane and polyvinyl chloride. These soft materials make the robot lighter, more compliant and difficult to control than other robots, which are traditionally stiff and rigid. Since this actuator works with vacuum, it is interesting to know at which negative pressure the actuator exerts which pulling force. It is also interesting to know at which contraction the most traction is exerted. Furthermore, the soft actuator is tested for material fatigue. A comparison shows that this actuator has a higher weight to force ratio than natural muscles. For the control of the artificial muscle, a control architecture is shown in this thesis.               
\begin{center}	
\normalsize \textbf{Zusammenfassung}\\
\end{center}
Hier die deutschsprachige Zusammenfassung. 
\optional{Talk to your supervisor if this is needed and/or wanted before starting with your thesis}
\end{abstract}
\newpage

%%%%%%%%%%%%%%%%%%%%% Widmung %%%%%%%%%%%%%%%%%%%%%%%%%%%%%%%%%
\phantom{u}
\phantom{1}\vspace{6cm}
\begin{center}
%Hier die Widmung oder leer lassen
\end{center}


\pagestyle{fancy}

%%%%%%%%%%%%%%%%%%%Inhaltsverzeichnis%%%%%%%%%%%%%%%%%%%%%%%%%%
\tableofcontents 

%%%%%%%%%%%%%%%%%%%%%%%%%%%%%%%
% ACTUAL CONTENT OF YOUR WORK %
%%%%%%%%%%%%%%%%%%%%%%%%%%%%%%%
%%%%%%%%%%%% Kapitel - externe Dateien zur Ordnung%%%%%%%%%%%%%
\ifLSRITRtutorial
	\input{./chapters/Tutorial.tex}
\fi
\input{./chapters/Introduction.tex}
\input{./chapters/MainPart.tex}
\input{./chapters/Evaluation.tex}
\input{./chapters/Discussion.tex}
\input{./chapters/Conclusion.tex}

\appendix
	\input{./chapters/Appendix.tex}
%%%%%%%%%%%%%%%%%%_Abbildungsverzeichnis %%%%%%%%%%%%%%%%%%%%%%
\cleardoublepage
\addcontentsline{toc}{chapter}{List of Figures} 
\listoffigures 	

%%%%%%%%%%%%%%%%%%_List of Tables %%%%%%%%%%%%%%%%%%%%%%
% A lit of tables can be added - if wanted and needed by commenting out the lines below
% \cleardoublepage
% \addcontentsline{toc}{chapter}{List of Tables} 
% \listoftables	

%%%%%%%%%%%%%%%%%%_Acronyms and Notations %%%%%%%%%%%%%%%%%%%%%%
\cleardoublepage
%ifdefined\AddMyGloss
\AddMyGloss 
% --> see include/gloss.tex
%\fi

%%%%%%%%%%%%%%%%%%Literaturverzeichnis %%%%%%%%%%%%%%%%%%%%%%%%
\cleardoublepage
\addcontentsline{toc}{chapter}{Bibliography}
\bibliography{./refs/mybib}
\bibliographystyle{alphaurl}

%%%%%%%%%%%%%%%%%%%%License %%%%%%%%%%%%%%%%%%%%%%%%%%%%%%%%%%%
\cleardoublepage
\chapter*{License}
\markright{LICENSE}
This work is licensed under the Creative Commons Attribution 3.0 Germany
License. To view a copy of this license,
visit \href{http://creativecommons.org/licenses/by/3.0/de/}{http://creativecommons.org} or send a letter
to Creative Commons, 171 Second Street, Suite 300, San
Francisco, California 94105, USA.

%%%%%%%%%%%%%%%%%%%%List of TODOs %%%%%%%%%%%%%%%%%%%%%%%%%%%%%%%%%%%
% this MUST be empty and removed in the final version of course!
\listoftodos
\end{document}
